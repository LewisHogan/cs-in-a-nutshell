\documentclass[12pt]{article}

% Inspired by the notes template available at https://github.com/dev-aditya/LaTeX-template/

\usepackage[dvipsnames]{xcolor} % Import a bunch of colours so we can use them for things like definitions or theorems.
\usepackage[many]{tcolorbox} % Used for making those colour boxes and definitions

\usepackage{amsthm}
\theoremstyle{definition}

\usepackage{listings} % Useful for code highlighting in various languages

% For new terms which might be unclear to the reader

\newenvironment{definition}{
    \textbf{Definition}
}
% \newtheorem{definition}[section]{Definition} % A version with the number
    
\tcolorboxenvironment{definition}{
    boxrule=0pt,
    boxsep=4pt, % Padding of the content within the box
    colback={White!90!Cerulean}, % 90% White, remaining is Cerulean for the background
    enhanced jigsaw, % Jigsaw removes the thing box around the edges.
    borderline west={2pt}{0pt}{Cerulean},
    sharp corners,
    before skip=10pt,
    after skip=10pt,
    breakable
}

% Notes that may help the user, even if they aren't introducing something new.
\newenvironment{note}{
    \textbf{Note}
}
% \newtheorem{note}{Note}

\tcolorboxenvironment{note}{
    boxrule=0pt,
    boxsep=4pt, % Padding of the content within the box
    colback={White!90!Yellow}, % 90% White, remaining is Cerulean for the background
    enhanced jigsaw, % Jigsaw removes the thing box around the edges.
    borderline west={2pt}{0pt}{Yellow},
    sharp corners,
    before skip=10pt,
    after skip=10pt,
    breakable
}

% Questions to test the reader
\newenvironment{question}{
    \textbf{Question}
}
% \newtheorem{question}{Danger}

\tcolorboxenvironment{question}{
    boxrule=0pt,
    boxsep=4pt,
    colback={White!90!Red},
    enhanced jigsaw,
    borderline west={2pt}{0pt}{Red},
    sharp corners,
    before skip=10pt,
    after skip=10pt,
    breakable
}

% Answers to the questions
\newenvironment{answer}{
    \textbf{Answer}
}
% \newtheorem{answer}{Solution}

\tcolorboxenvironment{answer}{
    boxrule=0pt,
    boxsep=4pt,
    colback={White!90!LimeGreen},
    enhanced jigsaw,
    borderline west={2pt}{0pt}{LimeGreen},
    sharp corners,
    before skip=10pt,
    after skip=10pt,
    breakable
}

\newtcbtheorem{tcbexercise}{Exercise}{
    width=\textwidth,
    colback=White!90!Yellow,
    colframe=Orange!90,
    colbacktitle=Orange!90,
    fonttitle=\bfseries,
    sharp corners,
    boxrule=1pt,
    breakable,
    enhanced,
    boxed title style={sharp corners},
    attach boxed title to top left
}{tcbexercise}

\newenvironment{exercise}[1][]{
    \begin{tcbexercise}{#1}{}
}{ \end{tcbexercise} }

\newtcbtheorem{tcbsolution}{Solution}{
    width=\textwidth,
    colback=White!90!LimeGreen,
    colframe=LimeGreen!90,
    colbacktitle=LimeGreen!90,
    fonttitle=\bfseries,
    sharp corners,
    boxrule=1pt,
    breakable,
    enhanced,
    boxed title style={sharp corners},
    attach boxed title to top left
}{tcbsolution}

\newenvironment{solution}[1][]{
    \begin{tcbsolution}{#1}{}
}{ \end{tcbsolution} }

\definecolor{codekeyword}{RGB}{182, 112, 203}
\definecolor{codecomment}{RGB}{42, 101, 73}
\definecolor{codeidentifier}{RGB}{0, 0, 0}
\definecolor{codestring}{RGB}{163, 21, 21}

\lstdefinestyle{code}{
    breaklines=true,
    numbers=left,
    xleftmargin=12pt,
    showstringspaces=false,
    basicstyle=\footnotesize\ttfamily,
    keywordstyle=\bfseries\color{codekeyword},
    commentstyle=\color{codecomment},
    identifierstyle=\color{codeidentifier},
    stringstyle=\color{codestring},
}

\lstset{style=code}

\author{Lewis Hogan}

\title{A sample document to show the preamble}
\begin{document}
    \maketitle
    \tableofcontents
    
    \section{Environments}

    \subsection{Definitions}
    \begin{definition}
        An environment is a structure used to format blocks of text in \LaTeX~documents.
    \end{definition}

    The definition environment is used to describe new and unfamiliar terms.
    
    \subsection{Notes}

    \begin{note}
        The syntax highlighting plugin being used for this preamble is called \emph{listings}    
    \end{note}

    The note environment is used for additional information that may be of interest to a reader, but should not be essential.

    \subsection{Question}

    \begin{question}
        What is the formula for Pythagoras' theorem?
    \end{question}

    The question environment is used for prompting the user to consider a point of view.

    \subsection{Answer}
    
    \begin{answer}
        The formula for Pythagoras's theorem is \[
          a^2+b^2=c^2  
        \]
        This is frequently used for calculating the \emph{hypotenuse} of any triangle with a right angle.
    \end{answer}

    The answer environment provides a solution for early prompted questions.

    \subsection{Danger}

    \begin{danger}
        To compile \LaTeX~documents you need to make sure you've installed all the required packages! 
    \end{danger}

    The danger environment provides warnings about common pitfalls which make cause issues for the reader.

    \subsection{Exercises}
    
    \begin{exercise}[Hello World!]
        \lstinputlisting[language=C++]{sample_incomplete.cpp}
    \end{exercise}
    Exercises can provide code and feature syntax highlighting.

    \subsection{Solutions}
    \begin{solution}[Hello World!]
        \lstinputlisting[language=C++]{sample_complete.cpp}
    \end{solution}

    Solutions provide the completed code for the corresponding exercise.
\end{document}